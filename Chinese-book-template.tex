\documentclass[]{ctexbook}
\usepackage{lmodern}
\usepackage{amssymb,amsmath}
\usepackage{ifxetex,ifluatex}
\usepackage{fixltx2e} % provides \textsubscript
\ifnum 0\ifxetex 1\fi\ifluatex 1\fi=0 % if pdftex
  \usepackage[T1]{fontenc}
  \usepackage[utf8]{inputenc}
\else % if luatex or xelatex
  \ifxetex
    \usepackage{xltxtra,xunicode}
  \else
    \usepackage{fontspec}
  \fi
  \defaultfontfeatures{Ligatures=TeX,Scale=MatchLowercase}
\fi
% use upquote if available, for straight quotes in verbatim environments
\IfFileExists{upquote.sty}{\usepackage{upquote}}{}
% use microtype if available
\IfFileExists{microtype.sty}{%
\usepackage{microtype}
\UseMicrotypeSet[protrusion]{basicmath} % disable protrusion for tt fonts
}{}
\usepackage[b5paper,tmargin=2.5cm,bmargin=2.5cm,lmargin=3.5cm,rmargin=2.5cm]{geometry}
\usepackage[unicode=true]{hyperref}
\PassOptionsToPackage{usenames,dvipsnames}{color} % color is loaded by hyperref
\hypersetup{
            pdftitle={中文书示例},
            colorlinks=true,
            linkcolor=Maroon,
            citecolor=Blue,
            urlcolor=Blue,
            breaklinks=true}
\urlstyle{same}  % don't use monospace font for urls
\usepackage{natbib}
\bibliographystyle{apalike}
\usepackage{color}
\usepackage{fancyvrb}
\newcommand{\VerbBar}{|}
\newcommand{\VERB}{\Verb[commandchars=\\\{\}]}
\DefineVerbatimEnvironment{Highlighting}{Verbatim}{commandchars=\\\{\}}
% Add ',fontsize=\small' for more characters per line
\usepackage{framed}
\definecolor{shadecolor}{RGB}{248,248,248}
\newenvironment{Shaded}{\begin{snugshade}}{\end{snugshade}}
\newcommand{\AlertTok}[1]{\textcolor[rgb]{0.94,0.16,0.16}{#1}}
\newcommand{\AnnotationTok}[1]{\textcolor[rgb]{0.56,0.35,0.01}{\textbf{\textit{#1}}}}
\newcommand{\AttributeTok}[1]{\textcolor[rgb]{0.77,0.63,0.00}{#1}}
\newcommand{\BaseNTok}[1]{\textcolor[rgb]{0.00,0.00,0.81}{#1}}
\newcommand{\BuiltInTok}[1]{#1}
\newcommand{\CharTok}[1]{\textcolor[rgb]{0.31,0.60,0.02}{#1}}
\newcommand{\CommentTok}[1]{\textcolor[rgb]{0.56,0.35,0.01}{\textit{#1}}}
\newcommand{\CommentVarTok}[1]{\textcolor[rgb]{0.56,0.35,0.01}{\textbf{\textit{#1}}}}
\newcommand{\ConstantTok}[1]{\textcolor[rgb]{0.00,0.00,0.00}{#1}}
\newcommand{\ControlFlowTok}[1]{\textcolor[rgb]{0.13,0.29,0.53}{\textbf{#1}}}
\newcommand{\DataTypeTok}[1]{\textcolor[rgb]{0.13,0.29,0.53}{#1}}
\newcommand{\DecValTok}[1]{\textcolor[rgb]{0.00,0.00,0.81}{#1}}
\newcommand{\DocumentationTok}[1]{\textcolor[rgb]{0.56,0.35,0.01}{\textbf{\textit{#1}}}}
\newcommand{\ErrorTok}[1]{\textcolor[rgb]{0.64,0.00,0.00}{\textbf{#1}}}
\newcommand{\ExtensionTok}[1]{#1}
\newcommand{\FloatTok}[1]{\textcolor[rgb]{0.00,0.00,0.81}{#1}}
\newcommand{\FunctionTok}[1]{\textcolor[rgb]{0.00,0.00,0.00}{#1}}
\newcommand{\ImportTok}[1]{#1}
\newcommand{\InformationTok}[1]{\textcolor[rgb]{0.56,0.35,0.01}{\textbf{\textit{#1}}}}
\newcommand{\KeywordTok}[1]{\textcolor[rgb]{0.13,0.29,0.53}{\textbf{#1}}}
\newcommand{\NormalTok}[1]{#1}
\newcommand{\OperatorTok}[1]{\textcolor[rgb]{0.81,0.36,0.00}{\textbf{#1}}}
\newcommand{\OtherTok}[1]{\textcolor[rgb]{0.56,0.35,0.01}{#1}}
\newcommand{\PreprocessorTok}[1]{\textcolor[rgb]{0.56,0.35,0.01}{\textit{#1}}}
\newcommand{\RegionMarkerTok}[1]{#1}
\newcommand{\SpecialCharTok}[1]{\textcolor[rgb]{0.00,0.00,0.00}{#1}}
\newcommand{\SpecialStringTok}[1]{\textcolor[rgb]{0.31,0.60,0.02}{#1}}
\newcommand{\StringTok}[1]{\textcolor[rgb]{0.31,0.60,0.02}{#1}}
\newcommand{\VariableTok}[1]{\textcolor[rgb]{0.00,0.00,0.00}{#1}}
\newcommand{\VerbatimStringTok}[1]{\textcolor[rgb]{0.31,0.60,0.02}{#1}}
\newcommand{\WarningTok}[1]{\textcolor[rgb]{0.56,0.35,0.01}{\textbf{\textit{#1}}}}
\usepackage{longtable,booktabs}
% Fix footnotes in tables (requires footnote package)
\IfFileExists{footnote.sty}{\usepackage{footnote}\makesavenoteenv{long table}}{}
\usepackage[normalem]{ulem}
% avoid problems with \sout in headers with hyperref:
\pdfstringdefDisableCommands{\renewcommand{\sout}{}}
\IfFileExists{parskip.sty}{%
\usepackage{parskip}
}{% else
\setlength{\parindent}{0pt}
\setlength{\parskip}{6pt plus 2pt minus 1pt}
}
\setlength{\emergencystretch}{3em}  % prevent overfull lines
\providecommand{\tightlist}{%
  \setlength{\itemsep}{0pt}\setlength{\parskip}{0pt}}
\setcounter{secnumdepth}{5}
% Redefines (sub)paragraphs to behave more like sections
\ifx\paragraph\undefined\else
\let\oldparagraph\paragraph
\renewcommand{\paragraph}[1]{\oldparagraph{#1}\mbox{}}
\fi
\ifx\subparagraph\undefined\else
\let\oldsubparagraph\subparagraph
\renewcommand{\subparagraph}[1]{\oldsubparagraph{#1}\mbox{}}
\fi

% set default figure placement to htbp
\makeatletter
\def\fps@figure{htbp}
\makeatother

\usepackage{booktabs}
\usepackage{longtable}

\usepackage{framed,color}
\definecolor{shadecolor}{RGB}{248,248,248}

\renewcommand{\textfraction}{0.05}
\renewcommand{\topfraction}{0.8}
\renewcommand{\bottomfraction}{0.8}
\renewcommand{\floatpagefraction}{0.75}

\let\oldhref\href
\renewcommand{\href}[2]{#2\footnote{\url{#1}}}

\makeatletter
\newenvironment{kframe}{%
\medskip{}
\setlength{\fboxsep}{.8em}
 \def\at@end@of@kframe{}%
 \ifinner\ifhmode%
  \def\at@end@of@kframe{\end{minipage}}%
  \begin{minipage}{\columnwidth}%
 \fi\fi%
 \def\FrameCommand##1{\hskip\@totalleftmargin \hskip-\fboxsep
 \colorbox{shadecolor}{##1}\hskip-\fboxsep
     % There is no \\@totalrightmargin, so:
     \hskip-\linewidth \hskip-\@totalleftmargin \hskip\columnwidth}%
 \MakeFramed {\advance\hsize-\width
   \@totalleftmargin\z@ \linewidth\hsize
   \@setminipage}}%
 {\par\unskip\endMakeFramed%
 \at@end@of@kframe}
\makeatother

\makeatletter
\@ifundefined{Shaded}{
}{\renewenvironment{Shaded}{\begin{kframe}}{\end{kframe}}}
\@ifpackageloaded{fancyvrb}{%
  % https://github.com/CTeX-org/ctex-kit/issues/331
  \RecustomVerbatimEnvironment{Highlighting}{Verbatim}{commandchars=\\\{\},formatcom=\xeCJKVerbAddon}%
}{}
\makeatother

\usepackage{makeidx}
\makeindex

\urlstyle{tt}

\usepackage{amsthm}
\makeatletter
\def\thm@space@setup{%
  \thm@preskip=8pt plus 2pt minus 4pt
  \thm@postskip=\thm@preskip
}
\makeatother

\frontmatter

\title{中文书示例}
\date{2021-04-01}

\begin{document}
\maketitle


\thispagestyle{empty}

\begin{center}
献给……

呃,爱谁谁吧
\end{center}

\setlength{\abovedisplayskip}{-5pt}
\setlength{\abovedisplayshortskip}{-5pt}

{
\setcounter{tocdepth}{2}
\tableofcontents
}
\listoftables
\listoffigures
\hypertarget{ux524dux8a00}{%
\chapter*{前言}\label{ux524dux8a00}}


第 \ref{markdown} 章是关于

\begin{flushright}
于 R 村某角落
\end{flushright}

\hypertarget{markdown}{%
\chapter{Markdown}\label{markdown}}

\texttt{Markdown}的基础语法:

\hypertarget{comment}{%
\section{Markdown 的注释}\label{comment}}

\begin{itemize}
\tightlist
\item
  文本注释:依然可使用Ctrl+Shift+C的快捷键,其结果为在待注释的文本前添加\texttt{\textless{}!-\/-},在其后添加\texttt{-\/-\textgreater{}}
\item
  R代码注释:无论是行内代码还是代码块,对于代码正文(不包括前后的斜引号及大括号里面的内容)来说,依然可使用Ctrl+Shift+C的快捷键,其结果与普通R脚本中的注释相同,为在待注释的文本前加井号\texttt{\#}
\end{itemize}

\hypertarget{ux6807ux9898ux548cux6bb5ux843d}{%
\section{标题和段落}\label{ux6807ux9898ux548cux6bb5ux843d}}

\hypertarget{ux6807ux9898-header}{%
\subsection{标题 (Header)}\label{ux6807ux9898-header}}

\begin{Shaded}
\begin{Highlighting}[]
\FunctionTok{\# 一级标题}
\FunctionTok{\#\# 二级标题}
\FunctionTok{\#\#\# 三级标题}
\FunctionTok{\#\#\#\# 四级标题}
\FunctionTok{\#\#\#\#\# 五级标题}
\FunctionTok{\#\#\#\#\#\# 六级标题}
\end{Highlighting}
\end{Shaded}

\hypertarget{ux6c34ux5e73ux7ebf-horizontal-rule}{%
\subsection{水平线 (Horizontal rule)}\label{ux6c34ux5e73ux7ebf-horizontal-rule}}

\begin{Shaded}
\begin{Highlighting}[]
\NormalTok{***}
\NormalTok{{-}{-}{-}}
\end{Highlighting}
\end{Shaded}

\hypertarget{ux5f3aux8c03-emphasize}{%
\subsection{强调 (Emphasize)}\label{ux5f3aux8c03-emphasize}}

\begin{itemize}
\item
  加粗和斜体

\begin{Shaded}
\begin{Highlighting}[]
\NormalTok{*这是斜体*}
\NormalTok{\_这也是斜体\_}
\NormalTok{**这是粗体**}
\NormalTok{\_\_这也是粗体\_\_}
\NormalTok{***这是粗体也是斜体***}
\NormalTok{\_\_\_这是粗体也是斜体\_\_\_}
\end{Highlighting}
\end{Shaded}

  \emph{这是斜体}

  \emph{这也是斜体}

  \textbf{这是粗体}

  \textbf{这也是粗体}

  \textbf{\emph{这是粗体也是斜体}}

  \textbf{\emph{这是粗体也是斜体}}
\item
  删除线

\begin{Shaded}
\begin{Highlighting}[]
\NormalTok{\textasciitilde{}\textasciitilde{}删除这句话\textasciitilde{}\textasciitilde{}}
\end{Highlighting}
\end{Shaded}

  \sout{删除这句话}
\end{itemize}

\hypertarget{ux6587ux5b57}{%
\subsection{文字}\label{ux6587ux5b57}}

\hypertarget{ux5b57ux4f53}{%
\subsubsection{字体}\label{ux5b57ux4f53}}

\begin{itemize}
\item
  \texttt{\textless{}span\ style="font-family:宋体;"\textgreater{}宋体\textless{}/span\textgreater{}}显示为{宋体},\\
\item
  \texttt{\textless{}span\ style="font-family:Times\ New\ Roman;"\textgreater{}Times\ New\ Roman\textless{}/span\textgreater{}}显示为{Times New
  Roman}
\item
  \texttt{\textless{}span\ style="font-variant:small-caps;"\textgreater{}Small\ Caps\textless{}/span\textgreater{}} 显示为 \textsc{Small Caps}
  \#\#\#\# 字号
\item
  \texttt{\textless{}span\ style="font-size:25px;"\textgreater{}字号25px\textless{}/span\textgreater{}}显示为 {字号25px}
\end{itemize}

\hypertarget{ux6587ux5b57ux989cux8272}{%
\subsubsection{文字颜色}\label{ux6587ux5b57ux989cux8272}}

\begin{itemize}
\tightlist
\item
  \texttt{\textless{}span\ style="color:red;"\textgreater{}红色文字\textless{}/span\textgreater{}}显示为{红色文字},
\item
  \texttt{\textless{}span\ style="color:\#33C0FF;"\textgreater{}文字色号\#33C0FF\textless{}/span\textgreater{}}显示为 {文字色号\#33C0FF}
\end{itemize}

\hypertarget{ux80ccux666fux989cux8272}{%
\subsubsection{背景颜色}\label{ux80ccux666fux989cux8272}}

\begin{itemize}
\tightlist
\item
  \texttt{\textless{}span\ style="background-color:yellow;"\textgreater{}背景为黄色\textless{}/span\textgreater{}}显示为{背景为黄色}
\item
  \texttt{\textless{}span\ style="background-color:\#33FF8B;"\textgreater{}背景色号为\#33FF8B\textless{}/span\textgreater{}}显示为{背景色号为\#33FF8B}
\end{itemize}

\hypertarget{ux5212ux7ebf}{%
\subsubsection{划线}\label{ux5212ux7ebf}}

\begin{itemize}
\item
  \texttt{\textless{}span\ style="text-decoration:\ underline;"\textgreater{}下划线\textless{}/span\textgreater{}}显示为{下划线}
\item
  \texttt{\textless{}span\ style="text-decoration:\ line-through;"\textgreater{}删除线\textless{}/span\textgreater{}}显示为{删除线}
\item
  \texttt{\textless{}span\ style="text-decoration:\ overline;"\textgreater{}上划线\textless{}/span\textgreater{}}显示为{上划线}
\end{itemize}

\hypertarget{ux5b57ux4f53ux989cux8272ux4e5fux53efux4ee5ux4f7fux7528fontux6807ux7b7eux5b9eux73b0.}{%
\subsubsection{\texorpdfstring{字体颜色也可以使用\texttt{\textless{}font\textgreater{}}标签实现.}{字体颜色也可以使用\textless font\textgreater 标签实现.}}\label{ux5b57ux4f53ux989cux8272ux4e5fux53efux4ee5ux4f7fux7528fontux6807ux7b7eux5b9eux73b0.}}

\texttt{\textless{}font\ color="blue"\textgreater{}这段话使用蓝色字体.\textless{}/font\textgreater{}} 显示为 这段话使用蓝色字体.

\texttt{\textless{}font\textgreater{}}标签常用属性:

\begin{longtable}[]{@{}ll@{}}
\toprule
属性 & 描述\tabularnewline
\midrule
\endhead
color & 颜色\tabularnewline
face & 字体\tabularnewline
size & 字体大小\tabularnewline
\bottomrule
\end{longtable}

\hypertarget{ux6bb5ux843d}{%
\subsection{段落}\label{ux6bb5ux843d}}

\begin{itemize}
\item
  一个段落由一行或连续的多行组成。
\item
  段落之间以空行分隔。
\item
  同一段落内的不同行在转换成HTML或docx等格式后会重新排列,原来的段内换行被当成了空格,这样的规定与LaTeX类似。
\item
  普通段落不该用空格或制表符来缩进, 不应在行尾留有空格。
\item
  为了在段内换行并且转化后仍保持段内换行, 输入时在前面行的末尾输入两个或两个以上空格。例如:

\begin{Shaded}
\begin{Highlighting}[]
\NormalTok{白日依山尽,黄河入海流。    }
\NormalTok{欲穷千里目,更上一层楼。}
\end{Highlighting}
\end{Shaded}

  显示为:

  白日依山尽,黄河入海流。\\
  欲穷千里目,更上一层楼。
\item
  这样做的缺点是末尾的空格是不可见的。 可以使用HTML的标签在段内换行,如:

\begin{Shaded}
\begin{Highlighting}[]
\NormalTok{白日依山尽,黄河入海流。}\KeywordTok{\textless{}br\textgreater{}}
\NormalTok{欲穷千里目,更上一层楼。。}
\end{Highlighting}
\end{Shaded}

  显示为:

  白日依山尽,黄河入海流。
  欲穷千里目,更上一层楼。
\end{itemize}

\hypertarget{ux6bb5ux843dux5bf9ux9f50ux548cux6bb5ux95f4ux8ddd}{%
\subsection{段落对齐和段间距}\label{ux6bb5ux843dux5bf9ux9f50ux548cux6bb5ux95f4ux8ddd}}

使用HTML\texttt{\textless{}p\textgreater{}}标签设置对齐和段间距.

\begin{itemize}
\item
  对齐
  对齐是\texttt{\textless{}p\textgreater{}}标签中的\texttt{align}属性.

  \begin{longtable}[]{@{}ll@{}}
  \toprule
  属性 & 描述\tabularnewline
  \midrule
  \endhead
  left & 左对齐\tabularnewline
  right & 右对齐\tabularnewline
  center & 居中对齐\tabularnewline
  justify & 两端对齐\tabularnewline
  \bottomrule
  \end{longtable}

\begin{Shaded}
\begin{Highlighting}[]
\KeywordTok{\textless{}p}\OtherTok{ align} \OtherTok{=} \StringTok{"left"}\KeywordTok{\textgreater{}} 
\KeywordTok{\textless{}/p\textgreater{}}
\end{Highlighting}
\end{Shaded}
\item
  行间距
  使用\texttt{\textless{}p\textgreater{}}标签的\texttt{style}属性进行设置.
  将\texttt{line-height}设置为\texttt{300\%}相当于是三倍行距.

\begin{Shaded}
\begin{Highlighting}[]
\KeywordTok{\textless{}p}\OtherTok{ align} \OtherTok{=} \StringTok{"justify"}\OtherTok{ style=}\StringTok{"line{-}height:300\%"}\KeywordTok{\textgreater{}} 
\KeywordTok{\textless{}/p\textgreater{}}
\end{Highlighting}
\end{Shaded}
\end{itemize}

\hypertarget{ux5f15ux7528blockquotes}{%
\subsection{引用(Blockquotes)}\label{ux5f15ux7528blockquotes}}

\begin{itemize}
\item
  引用也是段落模式,内容中的换行不起作用,空行导致分段。

\begin{Shaded}
\begin{Highlighting}[]
\AttributeTok{\textgreater{} 白日依山尽,黄河入海流。}
\AttributeTok{\textgreater{} 欲穷千里目,更上一层楼。}
\end{Highlighting}
\end{Shaded}

  转换成

  \begin{quote}
  白日依山尽,黄河入海流。
  欲穷千里目,更上一层楼。
  \end{quote}
\item
  引用段落也可以仅在段落第一行写大于号, 其它行顶格写,例如下面的两段引用:

\begin{Shaded}
\begin{Highlighting}[]
\AttributeTok{\textgreater{} 远上寒山石径斜,}
\AttributeTok{白云生处有人家。}
\AttributeTok{\textgreater{}}
\AttributeTok{\textgreater{} 停车坐爱枫林晚,}
\AttributeTok{霜叶红于二月花。。}
\end{Highlighting}
\end{Shaded}

  转换成

  \begin{quote}
  远上寒山石径斜,
  白云生处有人家。

  停车坐爱枫林晚,
  霜叶红于二月花。
  \end{quote}
\item
  为了在引用中换行,就需要加引用空行。为了排版诗、词之类的内容,希望人为控制换行和引导空格,可以将引用中的\texttt{\textgreater{}}替换成\texttt{\textbar{}},如:

\begin{Shaded}
\begin{Highlighting}[]
\NormalTok{| 白日依山尽,}
\NormalTok{| 黄河入海流。}
\NormalTok{| 欲穷千里目,}
\NormalTok{| 更上一层楼。}
\end{Highlighting}
\end{Shaded}

  转换成

  白日依山尽,\\
  黄河入海流。\\
  欲穷千里目,\\
  更上一层楼。
\item
  引用内也可以嵌套其它的Markdown格式如标题、列表等。 引用前后应该有空行把引用内容与其他内容分隔开。

\begin{Shaded}
\begin{Highlighting}[]
\AttributeTok{\textgreater{} 例如}
\AttributeTok{\textgreater{}}  
\AttributeTok{\textgreater{} (1) 顶层一;}
\AttributeTok{\textgreater{}     a) 内层1;}
\AttributeTok{\textgreater{}     b) 内层2;}
\AttributeTok{\textgreater{} (2) 顶层二。}
\end{Highlighting}
\end{Shaded}

  转换成

  \begin{quote}
  例如

  \begin{enumerate}
  \def\labelenumi{(\arabic{enumi})}
  \tightlist
  \item
    顶层一;

    \begin{enumerate}
    \def\labelenumii{\alph{enumii})}
    \tightlist
    \item
      内层1;
    \item
      内层2;
    \end{enumerate}
  \item
    顶层二。
  \end{enumerate}
  \end{quote}
\end{itemize}

\hypertarget{ux811aux6ce8}{%
\subsection{脚注}\label{ux811aux6ce8}}

Footnotes\footnote{This is a footnote.} are put inside the square brackets after a caret \texttt{\^{}{[}{]}}, e.g., \texttt{\^{}{[}This\ is\ a\ footnote.{]}}

\hypertarget{ux5217ux8868-list}{%
\section{列表 (List)}\label{ux5217ux8868-list}}

\begin{itemize}
\item
  有序列表 (Ordered list)

\begin{Shaded}
\begin{Highlighting}[]
\SpecialStringTok{1. }\NormalTok{Item 1}
\SpecialStringTok{1. }\NormalTok{Item 2}
\SpecialStringTok{1. }\NormalTok{Item 3}
\end{Highlighting}
\end{Shaded}
\item
  无序列表 (Unordered list)
  无需列表可以使用\texttt{*}或者\texttt{-}来创建.

\begin{Shaded}
\begin{Highlighting}[]
\SpecialStringTok{* }\NormalTok{Item 1}
\SpecialStringTok{* }\NormalTok{Item 2}
\SpecialStringTok{  * }\NormalTok{Item 2.1}
\SpecialStringTok{  * }\NormalTok{Item 2.2}
\SpecialStringTok{* }\NormalTok{Item 3}
\end{Highlighting}
\end{Shaded}

\begin{Shaded}
\begin{Highlighting}[]
\SpecialStringTok{{-} }\NormalTok{Item 1}
\SpecialStringTok{{-} }\NormalTok{Item 2}
\SpecialStringTok{  {-} }\NormalTok{Item 2.1}
\SpecialStringTok{  {-} }\NormalTok{Item 2.2}
\SpecialStringTok{{-} }\NormalTok{Item 3}
\end{Highlighting}
\end{Shaded}
\item
  任务列表 (Task list)
  有的markdown编辑器是支持任务列表的,比如github,但是有些也不支持.比如rmakdown就是不支持的(本书就是用rmakrdown写的).

\begin{Shaded}
\begin{Highlighting}[]
\SpecialStringTok{{-} }\VariableTok{[x]}\NormalTok{ Write the press release}
\SpecialStringTok{{-} }\VariableTok{[ ]}\NormalTok{ Update the website}
\SpecialStringTok{{-} }\VariableTok{[ ]}\NormalTok{ Contact the media}
\end{Highlighting}
\end{Shaded}
\item
  如果列表项目中有多个段落, 这时两个列表项之间应该以空行分隔, 每个项目除了第一行外,输入的每行内容都应该缩进4个空格或者一个制表符。
\end{itemize}

\hypertarget{ux94feux63a5-links}{%
\section{链接 (Links)}\label{ux94feux63a5-links}}

\begin{itemize}
\item
  \textbf{内部链接}
  还有一种链接是内部链接,用于文内跳转。 在各级标题行的末尾, 可以添加\{\#自定义标签\}这样的内容, 其中``自定义标签''是自己写的一个标识符, 标识符仅使用英文字母、数字、下划线、减号, 用来区分不同的位置。 比如,本文第一章添加了\texttt{markdown}为标签, 就可以用 \texttt{{[}回到第一章{]}(\#markdown)} 产生链接 \protect\hyperlink{markdown}{回到第一章}。
\item
  直接写url

\begin{Shaded}
\begin{Highlighting}[]
\NormalTok{https://guangyuzhu.rbind.io/}
\end{Highlighting}
\end{Shaded}

  \url{https://guangyuzhu.rbind.io/}
\item
  将网址超链接给某段文字

\begin{Shaded}
\begin{Highlighting}[]
\CommentTok{[}\OtherTok{我的网站}\CommentTok{](https://guangyuzhu.rbind.io/)}
\end{Highlighting}
\end{Shaded}

  \href{https://yylm.rbind.io/}{我的网站}
\item
  HTML插入链接
  使用\texttt{\textless{}a\textgreater{}}标签.

\begin{Shaded}
\begin{Highlighting}[]
\KeywordTok{\textless{}a}\OtherTok{ href=}\StringTok{"url"}\KeywordTok{\textgreater{}}\NormalTok{Link text}\KeywordTok{\textless{}/a\textgreater{}}
\end{Highlighting}
\end{Shaded}

  \begin{longtable}[]{@{}ll@{}}
  \toprule
  属性 & 描述\tabularnewline
  \midrule
  \endhead
  href & 链接网址url\tabularnewline
  target & 在何处打开链接, \texttt{\_blank}, \texttt{\_parent}, \texttt{\_self}, \texttt{\_top}\tabularnewline
  \bottomrule
  \end{longtable}

  比如我们插入一个网址,点击之后在新的网页打开.

\begin{Shaded}
\begin{Highlighting}[]
\KeywordTok{\textless{}a}\OtherTok{ href=}\StringTok{"https://guangyuzhu.rbind.io"}\OtherTok{ target} \OtherTok{=} \StringTok{\textquotesingle{}\_blank\textquotesingle{}}\KeywordTok{\textgreater{}}\NormalTok{我的网站}\KeywordTok{\textless{}/a\textgreater{}}\NormalTok{.}
\end{Highlighting}
\end{Shaded}

  我的网站.
\item
  将超链接绑定在图片上

\begin{Shaded}
\begin{Highlighting}[]
\KeywordTok{\textless{}a}\OtherTok{ href=}\StringTok{"https://guangyuzhu.rbind.io"}\OtherTok{ target} \OtherTok{=} \StringTok{\textquotesingle{}\_blank\textquotesingle{}}\KeywordTok{\textgreater{}}
\KeywordTok{\textless{}img}\OtherTok{ src} \OtherTok{=} \StringTok{\textquotesingle{}figures/avatar.jpeg\textquotesingle{}}\OtherTok{ alt} \OtherTok{=} \StringTok{\textquotesingle{}Guangyu Zhu\textquotesingle{}}\OtherTok{ height} \OtherTok{=} \StringTok{50\%}\OtherTok{ width} \OtherTok{=} \StringTok{50\%} \KeywordTok{/\textgreater{}}
\KeywordTok{\textless{}/a\textgreater{}}
\end{Highlighting}
\end{Shaded}

  点击下图查看我的网站:
\end{itemize}

\hypertarget{markdown-math}{%
\section{数学公式 (Equations)}\label{markdown-math}}

\begin{itemize}
\item
  Inline LaTeX equations can be written in a pair of dollar signs using the LaTeX syntax, e.g., \texttt{\$f(k)\ =\ \{n\ \textbackslash{}choose\ k\}\ p\^{}\{k\}\ (1-p)\^{}\{n-k\}\$} (actual output: \(f(k) = {n \choose k} p^{k} (1-p)^{n-k}\))
\item
  math expressions of the display style can be written in a pair of double dollar signs, e.g., \texttt{\$\$Y=X\textbackslash{}beta+\textbackslash{}epsilon\$\$}, and the output looks like this:
  \[Y=X\beta+\epsilon\]
\item
  用\texttt{\$\$}符号在两端界定的公式后面, 可以用\texttt{\textbackslash{}tag\{标号\}}命令增加人为的公式编号,如

\begin{Shaded}
\begin{Highlighting}[]
\NormalTok{$$}
\NormalTok{Y=X\textbackslash{}beta+\textbackslash{}epsilon \textbackslash{}tag\{1\}}
\NormalTok{$$}
\end{Highlighting}
\end{Shaded}

  结果显示为
  \[
  Y=X\beta+\epsilon \tag{1}
  \]
\end{itemize}

\begin{itemize}
\item
  要注意的是, \texttt{\textbackslash{}tag\{标号\}}命令一个公式只能用一个,这样多行的公式将不能为每行编号。例如,

\begin{Shaded}
\begin{Highlighting}[]
\NormalTok{$$f(k;p\_0\^{}*) = }
\NormalTok{\textbackslash{}begin\{cases\} p\_0\^{}* \& \textbackslash{}text\{if \}k=1, }\SpecialCharTok{\textbackslash{}\textbackslash{}}
\NormalTok{1{-}p\_0\^{}* \& \textbackslash{}text \{if \}k=0.\textbackslash{}end\{cases\}  \textbackslash{}tag\{2\}}
\NormalTok{$$}
\end{Highlighting}
\end{Shaded}

  结果显示为
  \[f(k;p_0^*) = 
  \begin{cases} p_0^* & \text{if }k=1, \\
  1-p_0^* & \text {if }k=0.\end{cases}  \tag{2}
  \]

  \begin{center}\rule{0.5\linewidth}{0.5pt}\end{center}

\begin{Shaded}
\begin{Highlighting}[]
\NormalTok{$$}
\NormalTok{\textbackslash{}begin\{aligned\} }
\NormalTok{  g(\textbackslash{}mu) \& = \textbackslash{}eta(x)}\SpecialCharTok{\textbackslash{}\textbackslash{}}
\NormalTok{  \textbackslash{}eta(X) \& = \textbackslash{}alpha+\textbackslash{}beta x}
\NormalTok{  \textbackslash{}end\{aligned\}  \textbackslash{}tag\{2\}}
\NormalTok{$$}
\end{Highlighting}
\end{Shaded}

  结果显示为
  \[
  \begin{aligned} 
    g(\mu) & = \eta(x)\\
    \eta(X) & = \alpha+\beta x
    \end{aligned}  \tag{2}
  \]

  \begin{center}\rule{0.5\linewidth}{0.5pt}\end{center}

\begin{Shaded}
\begin{Highlighting}[]
\NormalTok{$$}
\NormalTok{X = \textbackslash{}begin\{bmatrix\}1 \& x\_\{1\}}\SpecialCharTok{\textbackslash{}\textbackslash{}}
\NormalTok{1 \& x\_\{2\}}\SpecialCharTok{\textbackslash{}\textbackslash{}}
\NormalTok{1 \& x\_\{3\}  \textbackslash{}tag\{4\}}
\NormalTok{\textbackslash{}end\{bmatrix\}}
\NormalTok{$$}
\end{Highlighting}
\end{Shaded}

  结果显示为
  \[
  X = \begin{bmatrix}1 & x_{1}\\
  1 & x_{2}\\
  1 & x_{3}  \tag{4}
  \end{bmatrix}
  \]
\item
  用\texttt{\textbackslash{}tag}命令人为编号比较简单易用, 但是在有大量公式需要编号时就很不方便, 只要增加了一个公式就需要人为地重新编号并修改相应的引用。\texttt{bookdown}包支持对公式自动编号, 并可以按公式标签引用公式, 引用带有超链接。具体见 §\ref{bookdown-math}.
\end{itemize}

\hypertarget{ux4ee3ux7801ux5757-code-chunks}{%
\section{代码块 (Code chunks)}\label{ux4ee3ux7801ux5757-code-chunks}}

\begin{itemize}
\item
  行内代码

\begin{Shaded}
\begin{Highlighting}[]
\NormalTok{Add file with }\InformationTok{\textasciigrave{}git add\textasciigrave{}}
\end{Highlighting}
\end{Shaded}

  渲染结果:
  Add file with \texttt{git\ add}
\item
  单独代码

\begin{Shaded}
\begin{Highlighting}[]
\InformationTok{\textasciigrave{}\textasciigrave{}\textasciigrave{}bash}
\FunctionTok{git}\NormalTok{ status}
\FunctionTok{git}\NormalTok{ commit}
\InformationTok{\textasciigrave{}\textasciigrave{}\textasciigrave{}}
\end{Highlighting}
\end{Shaded}

  渲染结果:

\begin{Shaded}
\begin{Highlighting}[]
\FunctionTok{git}\NormalTok{ status}
\FunctionTok{git}\NormalTok{ commit}
\end{Highlighting}
\end{Shaded}
\item
  若代码中要显示斜引号本身,则需要用更多的斜引号括起来,最多含有n个不连续的斜引号要用n+1对斜引号括起来,n+1对斜引号与文本中间用空格分隔

  \begin{itemize}
  \tightlist
  \item
    为了显示\texttt{\textasciigrave{}\textasciigrave{}\textasciigrave{}code\textasciigrave{}\textasciigrave{}\textasciigrave{}},我们的代码应该为 \texttt{\textasciigrave{}\textasciigrave{}\textasciigrave{}\textasciigrave{}\ \textasciigrave{}\textasciigrave{}\textasciigrave{}code\textasciigrave{}\textasciigrave{}\textasciigrave{}\ \textasciigrave{}\textasciigrave{}\textasciigrave{}\textasciigrave{}}
  \item
    为了显示\texttt{\textasciigrave{}隔断\textasciigrave{}\textasciigrave{}中间文本\textasciigrave{}\textasciigrave{}隔断\textasciigrave{}},我们的代码应该为\texttt{\textasciigrave{}\textasciigrave{}\textasciigrave{}\ \textasciigrave{}隔断\textasciigrave{}\textasciigrave{}中间文本\textasciigrave{}\textasciigrave{}隔断\textasciigrave{}\ \textasciigrave{}\textasciigrave{}\textasciigrave{}}
  \end{itemize}
\item
  如果文本中没有斜引号,用多个斜引号括起来也可以,只要前后个数相同,且都在一行之内,效果与前后用一个斜引号相同。
\item
  为了显示
\end{itemize}

\begin{Shaded}
\begin{Highlighting}[]
\InformationTok{\textasciigrave{}\textasciigrave{}\textasciigrave{}bash}
\FunctionTok{git}\NormalTok{ push}
\InformationTok{\textasciigrave{}\textasciigrave{}\textasciigrave{}}
\end{Highlighting}
\end{Shaded}

代码为

\begin{Shaded}
\begin{Highlighting}[]
\InformationTok{\textasciigrave{}\textasciigrave{}\textasciigrave{}\textasciigrave{}markdown}
\InformationTok{\textasciigrave{}\textasciigrave{}\textasciigrave{}bash}
\FunctionTok{git}\NormalTok{ push}
\InformationTok{\textasciigrave{}\textasciigrave{}\textasciigrave{}}
\InformationTok{\textasciigrave{}\textasciigrave{}\textasciigrave{}\textasciigrave{}}
\end{Highlighting}
\end{Shaded}

\hypertarget{ux8868ux683c-tables}{%
\section{表格 (Tables)}\label{ux8868ux683c-tables}}

\begin{itemize}
\item
  Example 1

\begin{Shaded}
\begin{Highlighting}[]
\NormalTok{| Gender    | Age   | Major  |}
\NormalTok{| {-}{-}{-}{-}{-}{-}{-}{-}{-} |{-}{-}{-}{-}{-}{-}{-}| {-}{-}{-}{-}{-} |}
\NormalTok{| M         |18     | Statistics|}
\NormalTok{| F         | 19    |   Math |}
\end{Highlighting}
\end{Shaded}

  \begin{longtable}[]{@{}lll@{}}
  \toprule
  Gender & Age & Major\tabularnewline
  \midrule
  \endhead
  M & 18 & Statistics\tabularnewline
  F & 19 & Math\tabularnewline
  \bottomrule
  \end{longtable}
\item
  Example 2

\begin{Shaded}
\begin{Highlighting}[]
\CommentTok{{-}{-}{-}{-}{-}{-}{-}{-}{-}{-}{-}{-}{-}{-}{-}{-}{-}{-}{-}{-}{-}{-}{-}{-}{-}{-}{-}{-}{-}{-}{-}{-}{-}{-}{-}{-}{-}{-}{-}{-}{-}{-}{-}{-}{-}{-}{-}{-}{-}{-}{-}{-}{-}{-}{-}{-}{-}{-}{-}{-}{-}}
\CommentTok{ Centered   Default           Right Left}
\CommentTok{  Header    Aligned         Aligned Aligned}
\CommentTok{{-}{-}{-}}\NormalTok{{-}{-}{-}{-}{-}{-}{-}{-} {-}{-}{-}{-}{-}{-}{-} {-}{-}{-}{-}{-}{-}{-}{-}{-}{-}{-}{-}{-}{-}{-} {-}{-}{-}{-}{-}{-}{-}{-}{-}{-}{-}{-}{-}{-}{-}{-}{-}{-}{-}{-}{-}{-}{-}{-}{-}}
\NormalTok{   First    row                12.0 Example of a row that}
\NormalTok{                                    spans multiple lines.}

\NormalTok{  Second    row                 5.0 Here\textquotesingle{}s another one. Note}
\NormalTok{                                    the blank line between}
\NormalTok{                                    rows.}
\NormalTok{{-}{-}{-}{-}{-}{-}{-}{-}{-}{-}{-}{-}{-}{-}{-}{-}{-}{-}{-}{-}{-}{-}{-}{-}{-}{-}{-}{-}{-}{-}{-}{-}{-}{-}{-}{-}{-}{-}{-}{-}{-}{-}{-}{-}{-}{-}{-}{-}{-}{-}{-}{-}{-}{-}{-}{-}{-}{-}{-}{-}{-}}

\NormalTok{Table: Here\textquotesingle{}s the caption. It, too, may span}
\NormalTok{multiple lines.}
\end{Highlighting}
\end{Shaded}

  \begin{longtable}[]{@{}clrl@{}}
  \caption{Here's the caption. It, too, may span
  multiple lines.}\tabularnewline
  \toprule
  \begin{minipage}[b]{(\columnwidth - 3\tabcolsep) * \real{0.17}}\centering
  Centered
  Header\strut
  \end{minipage} & \begin{minipage}[b]{(\columnwidth - 3\tabcolsep) * \real{0.11}}\raggedright
  Default
  Aligned\strut
  \end{minipage} & \begin{minipage}[b]{(\columnwidth - 3\tabcolsep) * \real{0.22}}\raggedleft
  Right
  Aligned\strut
  \end{minipage} & \begin{minipage}[b]{(\columnwidth - 3\tabcolsep) * \real{0.36}}\raggedright
  Left
  Aligned\strut
  \end{minipage}\tabularnewline
  \midrule
  \endfirsthead
  \toprule
  \begin{minipage}[b]{(\columnwidth - 3\tabcolsep) * \real{0.17}}\centering
  Centered
  Header\strut
  \end{minipage} & \begin{minipage}[b]{(\columnwidth - 3\tabcolsep) * \real{0.11}}\raggedright
  Default
  Aligned\strut
  \end{minipage} & \begin{minipage}[b]{(\columnwidth - 3\tabcolsep) * \real{0.22}}\raggedleft
  Right
  Aligned\strut
  \end{minipage} & \begin{minipage}[b]{(\columnwidth - 3\tabcolsep) * \real{0.36}}\raggedright
  Left
  Aligned\strut
  \end{minipage}\tabularnewline
  \midrule
  \endhead
  \begin{minipage}[t]{(\columnwidth - 3\tabcolsep) * \real{0.17}}\centering
  First\strut
  \end{minipage} & \begin{minipage}[t]{(\columnwidth - 3\tabcolsep) * \real{0.11}}\raggedright
  row\strut
  \end{minipage} & \begin{minipage}[t]{(\columnwidth - 3\tabcolsep) * \real{0.22}}\raggedleft
  12.0\strut
  \end{minipage} & \begin{minipage}[t]{(\columnwidth - 3\tabcolsep) * \real{0.36}}\raggedright
  Example of a row that
  spans multiple lines.\strut
  \end{minipage}\tabularnewline
  \begin{minipage}[t]{(\columnwidth - 3\tabcolsep) * \real{0.17}}\centering
  Second\strut
  \end{minipage} & \begin{minipage}[t]{(\columnwidth - 3\tabcolsep) * \real{0.11}}\raggedright
  row\strut
  \end{minipage} & \begin{minipage}[t]{(\columnwidth - 3\tabcolsep) * \real{0.22}}\raggedleft
  5.0\strut
  \end{minipage} & \begin{minipage}[t]{(\columnwidth - 3\tabcolsep) * \real{0.36}}\raggedright
  Here's another one. Note
  the blank line between
  rows.\strut
  \end{minipage}\tabularnewline
  \bottomrule
  \end{longtable}
\item
  Example 3

\begin{Shaded}
\begin{Highlighting}[]
\NormalTok{+{-}{-}{-}{-}{-}{-}{-}{-}{-}{-}{-}{-}{-}{-}{-}+{-}{-}{-}{-}{-}{-}{-}{-}{-}{-}{-}{-}{-}{-}{-}+{-}{-}{-}{-}{-}{-}{-}{-}{-}{-}{-}{-}{-}{-}{-}{-}{-}{-}{-}{-}+}
\NormalTok{| Fruit         | Price         | Advantages         |}
\NormalTok{+===============+===============+====================+}
\NormalTok{| Bananas       | $1.34         | {-} built{-}in wrapper |}
\NormalTok{|               |               | {-} bright color     |}
\NormalTok{+{-}{-}{-}{-}{-}{-}{-}{-}{-}{-}{-}{-}{-}{-}{-}+{-}{-}{-}{-}{-}{-}{-}{-}{-}{-}{-}{-}{-}{-}{-}+{-}{-}{-}{-}{-}{-}{-}{-}{-}{-}{-}{-}{-}{-}{-}{-}{-}{-}{-}{-}+}
\NormalTok{| Oranges       | $2.10         | {-} cures scurvy     |}
\NormalTok{|               |               | {-} tasty            |}
\NormalTok{+{-}{-}{-}{-}{-}{-}{-}{-}{-}{-}{-}{-}{-}{-}{-}+{-}{-}{-}{-}{-}{-}{-}{-}{-}{-}{-}{-}{-}{-}{-}+{-}{-}{-}{-}{-}{-}{-}{-}{-}{-}{-}{-}{-}{-}{-}{-}{-}{-}{-}{-}+}
\end{Highlighting}
\end{Shaded}

  \begin{longtable}[]{@{}lll@{}}
  \toprule
  \begin{minipage}[b]{(\columnwidth - 2\tabcolsep) * \real{0.22}}\raggedright
  Fruit\strut
  \end{minipage} & \begin{minipage}[b]{(\columnwidth - 2\tabcolsep) * \real{0.22}}\raggedright
  Price\strut
  \end{minipage} & \begin{minipage}[b]{(\columnwidth - 2\tabcolsep) * \real{0.29}}\raggedright
  Advantages\strut
  \end{minipage}\tabularnewline
  \midrule
  \endhead
  \begin{minipage}[t]{(\columnwidth - 2\tabcolsep) * \real{0.22}}\raggedright
  Bananas\strut
  \end{minipage} & \begin{minipage}[t]{(\columnwidth - 2\tabcolsep) * \real{0.22}}\raggedright
  \$1.34\strut
  \end{minipage} & \begin{minipage}[t]{(\columnwidth - 2\tabcolsep) * \real{0.29}}\raggedright
  \begin{itemize}
  \tightlist
  \item
    built-in wrapper
  \item
    bright color
  \end{itemize}\strut
  \end{minipage}\tabularnewline
  \begin{minipage}[t]{(\columnwidth - 2\tabcolsep) * \real{0.22}}\raggedright
  Oranges\strut
  \end{minipage} & \begin{minipage}[t]{(\columnwidth - 2\tabcolsep) * \real{0.22}}\raggedright
  \$2.10\strut
  \end{minipage} & \begin{minipage}[t]{(\columnwidth - 2\tabcolsep) * \real{0.29}}\raggedright
  \begin{itemize}
  \tightlist
  \item
    cures scurvy
  \item
    tasty
  \end{itemize}\strut
  \end{minipage}\tabularnewline
  \bottomrule
  \end{longtable}
\end{itemize}

\hypertarget{ux56feux7247-image}\OtherTok{ width} \OtherTok{=} \StringTok{50\%} \KeywordTok{/\textgreater{}}
\end{Highlighting}
\end{Shaded}

  常用的一些参数:

  \begin{longtable}[]{@{}ll@{}}
  \toprule
  属性 & 描述\tabularnewline
  \midrule
  \endhead
  alt & 图像替代文本\tabularnewline
  src & 图片链接或者本地地址\tabularnewline
  height & 图片高度,单位可以是\%或者pixel\tabularnewline
  width & 图片宽度,单位可以是\%或者pixel\tabularnewline
  align & 图片位置,只支持left或者right\tabularnewline
  \bottomrule
  \end{longtable}
\item
  图片默认是靠左对齐.如果想要居中,需要使用\texttt{\textless{}p\textgreater{}}标签,也就是段落.

\begin{Shaded}
\begin{Highlighting}[]
\KeywordTok{\textless{}p}\OtherTok{ align} \OtherTok{=} \StringTok{"center"}\KeywordTok{\textgreater{}}
\KeywordTok{\textless{}img}\OtherTok{ src} \OtherTok{=} \StringTok{\textquotesingle{}\textquotesingle{}}\OtherTok{ alt} \OtherTok{=} \StringTok{\textquotesingle{}\textquotesingle{}}\OtherTok{ height} \OtherTok{=} \StringTok{50\%}\OtherTok{ width} \OtherTok{=} \StringTok{50\%} \KeywordTok{/\textgreater{}}
\KeywordTok{\textless{}/p\textgreater{}}
\end{Highlighting}
\end{Shaded}
\end{itemize}

\hypertarget{ux63d2ux5165icon}{%
\section{插入icon}\label{ux63d2ux5165icon}}

可以插入符号工具包\texttt{Font\ Awesome}.一般的markdown编辑器都已经支持直接使用HTML插入icon.

比如插入微信图标:

\begin{Shaded}
\begin{Highlighting}[]
\KeywordTok{\textless{}i}\OtherTok{ class=}\StringTok{"fa fa{-}weixin"}\KeywordTok{\textgreater{}\textless{}/i\textgreater{}}
\end{Highlighting}
\end{Shaded}

如果不支持,可以在markdown的最底端插入下面的代码:

\begin{Shaded}
\begin{Highlighting}[]
\KeywordTok{\textless{}head\textgreater{}} 
    \KeywordTok{\textless{}script}\OtherTok{ defer src=}\StringTok{"https://use.fontawesome.com/releases/v5.0.13/js/all.js"}\KeywordTok{\textgreater{}\textless{}/script\textgreater{}} 
    \KeywordTok{\textless{}script}\OtherTok{ defer src=}\StringTok{"https://use.fontawesome.com/releases/v5.0.13/js/v4{-}shims.js"}\KeywordTok{\textgreater{}\textless{}/script\textgreater{}} 
\KeywordTok{\textless{}/head\textgreater{}} 
\KeywordTok{\textless{}link}\OtherTok{ rel=}\StringTok{"stylesheet"}\OtherTok{ href=}\StringTok{"https://use.fontawesome.com/releases/v5.0.13/css/all.css"}\KeywordTok{\textgreater{}}
\end{Highlighting}
\end{Shaded}

当然也可以使用HTML的\texttt{\textless{}a\textgreater{}}标签让icon绑定一个超链接.

\begin{Shaded}
\begin{Highlighting}[]
\KeywordTok{\textless{}a}\OtherTok{ href=}\StringTok{"https://guangyuzhu.rbind.io"}\KeywordTok{\textgreater{}\textless{}i}\OtherTok{ class=}\StringTok{"fa fa{-}weixin"}\KeywordTok{\textgreater{}\textless{}/i\textgreater{}\textless{}/a\textgreater{}}
\end{Highlighting}
\end{Shaded}

\hypertarget{ux89c6ux9891-videoux548cux97f3ux4e50-music}{%
\section{视频 (Video)和音乐 (music)}\label{ux89c6ux9891-videoux548cux97f3ux4e50-music}}

\hypertarget{ux672cux5730ux89c6ux9891}{%
\subsection{本地视频}\label{ux672cux5730ux89c6ux9891}}

\begin{Shaded}
\begin{Highlighting}[]
\KeywordTok{\textless{}div}\OtherTok{ class=}\StringTok{"embed{-}responsive embed{-}responsive{-}16by9"}\KeywordTok{\textgreater{}}
\KeywordTok{\textless{}video}\OtherTok{ width=}\StringTok{"70\%"}\OtherTok{ controls}\KeywordTok{\textgreater{}}
  \KeywordTok{\textless{}source}\OtherTok{ src=}\StringTok{"videos/Cherry Blossom.mp4"}\OtherTok{ type=}\StringTok{"video/mp4"}\KeywordTok{\textgreater{}}
\KeywordTok{\textless{}/video\textgreater{}}
\KeywordTok{\textless{}/div\textgreater{}}
\end{Highlighting}
\end{Shaded}

\hypertarget{ux672cux5730ux97f3ux9891}{%
\subsection{本地音频}\label{ux672cux5730ux97f3ux9891}}

\begin{Shaded}
\begin{Highlighting}[]
\KeywordTok{\textless{}figure\textgreater{}}
    \CommentTok{\textless{}!{-}{-}\textless{}figcaption\textgreater{}Listen to :\textless{}/figcaption\textgreater{}{-}{-}\textgreater{}}
    \KeywordTok{\textless{}audio}\OtherTok{ controls loop autoplay}\KeywordTok{\textgreater{}}
        \KeywordTok{\textless{}source}\OtherTok{ src=}\StringTok{"audios/Primrose, the Dancer.m4a"}\OtherTok{  type=}\StringTok{"audio/mp3"}\KeywordTok{\textgreater{}}\NormalTok{\textgreater{}}
    \KeywordTok{\textless{}/audio\textgreater{}}
\KeywordTok{\textless{}/figure\textgreater{}}   
\end{Highlighting}
\end{Shaded}

\begin{quote}
\end{quote}

\hypertarget{embed-audio-from-soundcloud}{%
\subsection{Embed audio from soundcloud}\label{embed-audio-from-soundcloud}}

\href{https://soundcloud.com/}{Soundcloud}

\begin{Shaded}
\begin{Highlighting}[]
\CommentTok{\textless{}!{-}{-} I changed the code to make it minimal {-}{-}\textgreater{}}
\KeywordTok{\textless{}iframe}\OtherTok{ width=}\StringTok{"50\%"}\OtherTok{ height=}\StringTok{"66"} 
\OtherTok{        scrolling=}\StringTok{"no"}\OtherTok{ frameborder=}\StringTok{"no"} 
\OtherTok{        allow=}\StringTok{"autoplay"} 
\OtherTok{        src=}\StringTok{"https://w.soundcloud.com/player/?url=https\%3A//api.soundcloud.com/tracks/471669768}\ErrorTok{\&}\StringTok{color=\%23ff5500}\ErrorTok{\&}\StringTok{auto\_play=false}\ErrorTok{\&}\StringTok{hide\_related=true}\ErrorTok{\&}\StringTok{show\_comments=false}\ErrorTok{\&}\StringTok{show\_user=false}\ErrorTok{\&}\StringTok{show\_reposts=false}\ErrorTok{\&}\StringTok{show\_teaser=false"}\KeywordTok{\textgreater{}}
\KeywordTok{\textless{}/iframe\textgreater{}}
\end{Highlighting}
\end{Shaded}

\hypertarget{ux6dfbux52a0ux9644ux4ef6}{%
\section{添加附件}\label{ux6dfbux52a0ux9644ux4ef6}}

用HTML语法实现:

\begin{Shaded}
\begin{Highlighting}[]
\KeywordTok{\textless{}a}\OtherTok{ href=}\StringTok{"data/AZT.Rdata"}\OtherTok{ download=}\StringTok{"AZT.Rdata"}\KeywordTok{\textgreater{}}\NormalTok{点击下载附件 *AZT.Rdata*}\KeywordTok{\textless{}/a\textgreater{}}
\end{Highlighting}
\end{Shaded}

效果为

点击下载附件 \emph{AZT.Rdata}

\bibliography{book.bib,packages.bib}

\backmatter
\printindex

\end{document}
